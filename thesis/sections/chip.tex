\section{Specyfikacja CHIP-8}
 CHIP-8 jest to interpretowany język programowania oryginalnie zaprojektowany dla komputerów DIY późnych lat 70 i wczesnych 80. Jego główną ideą było uczynienie procesu tworzenia gier łatwiejszym, a także danie możliwości przenoszenia kodu na różne maszyny. W tym celu opracowane zostało specjalne środowisko uruchomieniowe nazywane maszyną wirtualną. Jego schemat jest ogólnodostępny i dobrze udokumentowany. Język ten nie odniósł większego sukcesu jako stricte komputerowy. Stał on się przede wszystkim głównym oprogramowaniem dla mniejszych urządzeń w latach 80. i 90. szczególnie dla kalkulatorów. Najbardziej znanym modelem jest \textit{TI-83}, wydany przez \textit{Texas Instruments}, do zaawansowanych obliczeń naukowych \cite{Cowgod}.

\subsection{Pamięć}
 Język CHIP-8 jest kompatybilny z pamięcią \textit{RAM} do 4KB (4086B), której lokalizacja zaczyna się od adresu 0x000 (0) i kończy na 0xFFF (4096). Pierwsze 512 (0x000 - 0x1FF) bajtów odnoszą się do interpretera i nie są używane przez żadne inne programy, które w większości wczytywane są do pamięci od następnego bajtu (0x200), choć zdarzają się również takie, które wczytywane są dopiero przy 1536 (0x600) bajcie.

\subsection{Rejestry}
  CHIP-8 posiada 8-bitowe rejestry ogólnego przeznaczenia. Ich nazwy zaczynają się od \textit{V}, a następnie jest im przypisana kolejna cyfra heksadecymalna, co daje szesnaście rejestrów (V0-VF), z tym, że rejestr VF jest używany jako flaga dla niektórych instrukcji, dlatego nie powinien być wykorzystywany. Przygotowano także 16-bitowy rejestr nazywany \textit{I} służący głównie do przechowywania adresów pamięci. Warto wspomnieć o dwóch 8-bitowych rejestrach specjalnego przeznaczenia są one wykorzystywane do trzymania stanu dla opóźniacza (ang. delay) i czasomierza. Są one automatycznie obniżane co 60Hz. Istnieją jeszcze tak zwane \textit{pseduo-rejestry} do których nie ma dostępu z wykonywanych programów. Jeden z nich \textit{program counter} (16-bitowy), trzyma adres obecnie wykonywanej komórki w pamięci. Kolejny, \textit{stack pointer} (8-bitowy) jest używany do wskazywania najwyższego poziomu kopca. Natomiast kopiec to tablica szesnastu elementów po 16-bitów, w której trzymane są adresy. Powinny one zostać zwrócone po zakończeniu danego podprogramu (eng. \textit{subroutine}). CHIP-8 pozwala na działanie do szesnastu zagnieżdżonych podprogramów. 

\subsection{Klawiatura}
Układ klawiszy obsługiwanych przez maszynę wirtualną CHIP-8, odpowiada temu wykorzystanemu w komputerze \textit{COSMAC VIP}. Posiada on klawiaturę heksadecymalną, której schemat znajduje się poniżej:
\begin{table}[h!]
 \centering
 \caption{Układ klawiszy CHIP-8}
 \label{C8Keyboard}
 \begin{tabular}{|c|c|c|c|}
   \hline 
   1 & 2 & 3 & C\\
   \hline
   4 & 5 & 6 & D\\
   \hline
   7 & 8 & 9 & E\\
   \hline
   A & 0 & B & F\\
   \hline
 \end{tabular} 
\end{table}
\newpage
\subsection{Ekran}
 W oryginalnej implementacji języka CHIP-8 został użyty monochromatyczny ekran o rozmiarze 64x32-piksele. Każdy piksel identyfikowany jest za pomocą współrzędnych \textit{x}, \textit{y}, których numeracja zaczyna się w lewym górnym rogu ekranu. 

\begin{table}[h!]
  \centering
  \caption{Format ekranu}
  \label{C8Display}
  \begin{tabular}{|c c|}
    \hline 
    (0, 0) & (63, 0)\\ [1ex] 
    (0, 31) & (63, 31)\\ 
    \hline
  \end{tabular} 
\end{table}

Obraz rysowany jest poprzez tak zwane duszki (ang. \textit{sprite}), czyli grupie bajtów odpowiadających binarnej reprezentacji obrazu. Mogą one mieć wielkość do 15 bajtów, co przekłada się na maksymalny rozmiar 8x15 pikseli.\\

Dodatkowo programy wczytane do pamięci urządzenia mogą odnosić się do duszków reprezentujących notacje szesnastkową, czyli znaków od \textit{0} do \textit{F}. Ich dane w postaci heksadecymalnej powinny znajdować się w obszarze pamięci interpretera (adresy od 0x000 do 0x1FF), po włączeniu urządzenia.


\subsection{Dźwięk i zegary}
 W swojej pierwotnej implementacji komputer posiada dwa zegary, jeden służy do opóźnień (ang. \textit{delay timer}), a drugi do sterowania dźwiękiem. Oba zegary aktywne są tylko wtedy, gdy ich rejestry mają wartość większą od zera. Zadaniem pierwszego z nich jest jedynie odejmowanie jedynki z wartości trzymanej w rejestrze DT przy częstotliwości 60Hz. Drugi natomiast robi dokładnie to samo, również odejmując jeden od wartości w rejestrze ST przy dokładnie tej samej częstotliwości. Różnica polega na tym, że w chwili dekrementacji wartości trzymanej w rejestrze, zegar ten uruchamia zamontowany w komputerze brzęczyk. Ma on tylko jeden ton, a jego zakres nie jest nigdzie podany i zależy wyłącznie od autora interpretera.
 
\subsection{Instrukcje procesora}
Język CHIP-8 posiada 36 różnych instrukcji odnoszących się do działań matematycznych, obsługi grafiki i sterowania wykonywaniem programu. Każda z nich składa się z dwóch bajtów, z czego ten najbardziej znaczący musi być przechowywany w parzyście zaadresowanej komórce pamięci \textit{RAM}. Bajty ładowanych duszów muszą być dopełnione tak, aby nie złamać tej zasady. Lista wszystkich dostępnych kodów procesora wraz z ich opisami jest dostępna w instrukcji komputera \textit{COSMAC VIP} \cite{COSMAC} lub w \textit{Cowgod’s Chip-8 Technical Reference} \cite{Cowgod}.