\section{Komputery 8-bitowe}
\subsection{Architektura 8-bitowa}
To architektura komputera, w której używa się 8-bitowych rejestrów, jednostek \\arytmetyczno-logicznych i komórek pamięci \cite{Stakem}. Ta ilość danych była jednak niewystarczająca przy adresowaniu pamięci, dlatego komputery te posiadały najczęściej 16-bitową \textit{szynę pamięci} \cite{Stakem}, która pozwala na indeksowanie komórek do \textit{64kB~\footnote{Pamięć była maksymalnie adresowana od 0 do $2^{16}$ czyli 65536 bitów.}}. Wówczas rozszerzano, również wielkość instrukcji i niektórych rejestrów do wartości 16-bitowych.

\subsection{Historia}
Rozwój komputerów rozpoczął się na dobre po drugiej wojnie światowej. W 1947 roku został wynaleziony tranzystor bipolarny, który od 1955 roku zaczął zastępować duże lampy elektronowe, znacząco zmniejszając rozmiary komputerów. Odtąd tranzystory mogły posiadać tysiące bramek logicznych przy relatywnie małym zużyciu miejsca. Następnym wielkim odkryciem, było wprowadzenie układów scalonych. Jack Kilby opracował pierwszy działający czip. Pół roku później Robert Noyce \cite{Petzold}, zaprezentował swój układ, w którym rozwiązał wiele problemów jego konkurenta, największą zmianą było zastosowanie krzemu, zamiast germanu. Rozpoczęły się wyścigi firm technologicznych, aby stworzyć jak mniej zasobożerne jednostki. W 1968 roku Noyce wraz z Gordon E. Moore'em (późniejszym twórcą prawa Moore'a) otwierają firmę \textit{Intel}, zaledwie trzy lata później wypuszczają na rynek pierwszy komercyjny \cite{Stakem} mikroprocesor \textit{Intel 4004}, który z uwagi na małą moc obliczeniową trafia głównie do kalkulatorów \cite{Petzold}, jednak jego twórcy pozostawiają innym możliwość jego przeprogramowania. Dało możliwość pierwszym hobbistom elektroniki na wykorzystanie tych czipów we własnym zakresie. Dopiero drugi produkt \textit{Intela} przyniósł rewolucję. Zrezygnowano z architektury 4-bitowej na rzecz 8-bitowej, która w tamtych czasach była jedynie opracowana teoretycznie. \textit{Intel 8008} ugruntował pozycje firmy i przyniósł ogromne zyski. Pierwotnie procesor był zaprojektowany dla firmy \textit{CTC}, która chciała ich użyć do swoich terminali. \textit{Datapoint 2200} \cite{Stakem} miał służyć do odczytywania danych z jednostek typu mainframe, czyli ogromnych stacji służących do krytycznych obliczeń finansowych lub statystycznych. Dzięki wykorzystaniu charakterystyki dostarczonej przez swoich  partnerów, firma z Santa Clara opracowała technologię, która mogła napędzić pierwsze komputery osobiste. \\

 Zanim jeszcze zaczęto sprzedawać pierwsze produkty \textit{Intela}, Joseph Weisbecker w 1971 roku pracował przy projekcie procesora 8-bitowego \cite{Edwards}, który został przerwany po premierze \textit{i4004}. Jednostka sprzedawała się na tyle dobrze, że firma w której pracował Weisbecker, RCA zgodziła na kontynuowanie prac nad własnym czipem. Niedługo potem, zaprezentowano pierwszy procesor tej firmy, stworzony w technologii \textit{CMOS} (ang. \textit{Complementary Metal-Oxide Semiconductor}), złożony z dwóch układów \textit{COSMAC 1801U} i \textit{1801R} \cite{Edwards}, rok później połączono je w jeden chip o nazwie \textit{COSMAC 1802} . W między czasie (1975 rok), Weisbecker wydał zestaw edukacyjny pod nazwą RCA Microtut, zestaw do samodzielnego złożenia (ang. \textit{Do-It-Yourself Computer}), który miał uczyć podstaw budowy komputera i programowania. Niedługo później \textit{RCA} wydało konsole \textit{RCA Studio II}, powstałą na wniosek Weisbeckera. Opierał się na wcześniej wspominanym czipie \textit{COSMAC 1802}. Komputer nie odniósł większego sukcesu, prawdopodobnie z uwagi na brak kontrolerów, zamiast, których użyto klawiatury heksadecymalnej \cite{Cowgod}. Sprzęt ten posłużył córce Josepha, Joyce Weisbecker, która nauczyła się programować dzięki tej konsoli. W historii zapisała się jako pierwsza kobieta, która tworzyła gry komercyjne \cite{Edwards}. W 1976 roku czasopismo \textit{Popular Electronics} wydało opracowanych przez Weisbeckera schemat budowy komputera \textit{COSMAC ELF}, będącą blisko powiązaną z \textit{Microtut}jednostką zaprojektowaną dla hobbistów, którzy bez większych przeszkód mogli złożyć go w swoim domu. W późniejszej serii magazyn dodał do urządzenia prostą kartę graficzną opartą o tę z konsol \textit{Studio II} \cite{Edwards}. W tym samym roku \textit{RCA}, oddało do sprzedaży następny komputer, \textit{COSMAC VIP}, który w głównej mierze opierał się na \textit{COSMAC ELF}. Posiadał on wbudowany język programowania zaprojektowany przez Weisbeckera - CHIP-8 \cite{Cowgod}.
 

