\section{Wstęp}
W dobie błyskawicznego rozwoju internetu i komputerów, maszyny potrzebne do przetwarzania dużych porcji informacji są coraz bardziej skomplikowane. Z jednej strony, ważne są jak najmniejsze rozmiary tych urządzeń, z drugiej ceni się ich wydajność. To prowadzi do tworzenia coraz bardziej skomplikowanych podzespołów i systemów je napędzających. Przeciętna instrukcja komputera z lat 80. zajmowała około 150 stron\footnote{Instrukcja dla programistów najpopularniejszego komputera lat 80 - Commodore 64 zawiera 491 strony \cite{Commodore}.}, a jej zawartość opisywała budowę i działanie urządzenia, włącznie z zaprezentowaniem wszystkich instrukcji wbudowanego języka programowania, wyświetlaniem i manipulacją grafiki, czy dźwięku. Dzisiejsze instrukcje dla samych procesorów zawierają 5000 stron\footnote{Instrukcja dla projektantów oprogramowania, procesora Intel o architekturze i32 zawiera 4670 stron \cite{Intel}.}, a jest to jeden z elementów jednostki centralnej, poza tym pozostaje jeszcze system. Z tego powodu współczesne tworzenie oprogramowania opiera się na jeszcze większej abstrakcji, zestawach bibliotek i narzędzi. Nie mniej znajomość działania procesora, czy pamięci komputera, nadal stanowi istotną kwestie, dla twórców oprogramowania i inżynierów. Pod warstwą obecnie dostarczanych, modułów i pakietów, nadal są przeprowadzane typowe dla wczesnych komputerów operacje. Stanowią one dobre źródło do tego, aby zagłębić się w budowę współczesnych architektur. \\

Projekt ten powstał w celach hobbistycznych, jako wynik fascynacji 8-bitowymi komputerami i chęci zbudowania własnej jednostki. Jego założeniem jest przybliżenie użytkownikowi działania prostej architektury wczesnych elektronicznych maszyn cyfrowych. Dlatego został on upubliczniony jako \textit{wolne oprogramowanie} w serwisie \textit{github \footnote{Adres do repozytorium: github.com/krskibin/pyChip8 }} na \textit{licencji MIT\footnote{Więcej o licencji \textit{MIT} na stronie: https://opensource.org/licenses/MIT}}. W jego ramach było przygotowanie emulatora maszyny wirtualnej CHIP-8. Opracowany program działa jako aplikacja okienkowa, która odtwarza napisane w tym języku pliki binarne, wraz z emulacją grafiki, dźwięku i obsługą klawiatury. \\

W rozdziale drugim została poruszona kwestia komputerów 8-bitowych, ich historia, w tym również omawianej maszyny wirtualnej CHIP-8. Następnie w rozdziale trzecim przedstawiono zagadnienia teoretyczne związane z pracą. Na początku omówiono maszynę wirtualną, związane z nią zagadnienia, takie jak emulacja, czy wirtualizacją, przedstawiono różnicę między tymi pojęciami, następnie opisano zagadnienia potrzebne do opracowania własnego emulatora, takie jak przesunięcia bitowe, stosy, czy rejestry. Rozdział czwarty został poświęcony specyfikacji omawianego CHIP-8, jest to abstrakcyjna instrukcja, którą musiał spełniać komputer, na którym chciano zaimplementować ten system. W tym rozdziale poruszono budowę jego pamięci, procesora, wspieranych urządzeń wejścia, wyjścia. W rozdziale piątym przedstawiono specyfikację projektu, to jakie problemy niesie za sobą budowa emulatora, jego architekturę, strukturę klas, a także wymagania funkcyjne i niefunkcyjne, wraz z schematami użycia stworzonego programu. Następnie w rozdziale szóstym zawarto implementacje projektu, opisano proces budowy poszczególnych elementów maszyny wirtualnej, a także przedstawiono implementacje niektórych z instrukcji procesora. Rozdział siódmy poświęcony był napisanym testom. Pokrótce zaprezentowana została ich struktura i omówiona najważniejsze techniki odpowiedzialne za ich poprawne przygotowanie. Rozdział ósmy, to z kolei, podsumowanie pracy, a także przedstawienie kolejnych etapów jej rozwoju.